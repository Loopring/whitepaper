% !TEX program = XeLaTeX
% !TEX encoding = UTF-8
\documentclass[UTF8,nofonts]{article}
%{ctexart}


%\setCJKmainfont[BoldFont=FandolSong-Bold.otf,ItalicFont=FandolKai-Regular.otf]{FandolSong-Regular.otf}
%\setCJKsansfont[BoldFont=FandolHei-Bold.otf]{FandolHei-Regular.otf}
%\setCJKmonofont{FandolFang-Regular.otf}



\usepackage[T1]{fontenc} % recommended for languages with accents
\renewcommand{\refname}{Referencias}
\renewcommand{\figurename}{Figura}



\usepackage{url}
\usepackage{cancel}
\usepackage{xspace}
\usepackage{graphicx}
\usepackage{multicol}
\usepackage{multirow}
\usepackage{subfig}
\usepackage{amsmath}
\usepackage{amssymb}
\usepackage[a4paper, width=186mm, top=18mm, bottom=18mm, includeheadfoot]{geometry}
%\usepackage[a4paper, width=140mm, top=18mm, bottom=22mm, includeheadfoot]{geometry}
\usepackage{booktabs}
\usepackage{array}
\usepackage{verbatim}
\usepackage{caption}
\usepackage{natbib}
\usepackage{booktabs}
\usepackage{float}
\usepackage{pdflscape}
\usepackage{mathtools}
\usepackage[usenames, dvipsnames]{xcolor}
\usepackage{afterpage}
\usepackage{pgf}
\usepackage{tikz}
\usepackage{dirtree}
\usepackage[style=american]{csquotes}
\usepackage{amsfonts}
\usepackage{tikz}
\usepackage{tkz-graph}
\usetikzlibrary{arrows,decorations.pathmorphing,automata,positioning,backgrounds,fit,shapes.symbols,chains,intersections}

\newtheorem{definition}{Definition}[section]
\newtheorem{theorem}{Theorem}[section]
\newtheorem{lemma}{Lemma}
\newtheorem{proof}{Proof} [section]



\usepackage[toc, page, title, titletoc, header]{appendix}
\usepackage{marginnote}
\usepackage{tablefootnote}

%\renewcommand\appendixname{附\ 录}
%\renewcommand\appendixpagename{附\ 录}
%\renewcommand\appendixtocname{附\ 录}
\renewcommand\abstractname{Abstracto}


\usepackage{perpage} %the perpage package
\MakePerPage{footnote} %the perpage package command

\usetikzlibrary{shapes.geometric}%
\usepackage{color}
%\usepackage[pages=some, placement=top]{background}
\usepackage{eso-pic}
\usepackage[final]{pdfpages}

%\includepdf[pages=1]{cover}
\hyphenpenalty=750

\title{\textbf{Loopring:}\\\textbf{Un Protocolo Descentralizado de Plataformas de Intercambio de Tokens}}
\author{
  Daniel Wang\\
  \texttt{daniel@loopring.org}\\
  \and
  	Jay Zhou\\
  	\texttt{jay@loopring.org}\\
  	\and
  	Alex Wang\\
  	\texttt{alex@loopring.org}\\
  	\and
  	Matthew Finestone\\
  	\texttt{matt.finestone@gmail.com}\\ 
  \\
  \texttt{https://loopring.org}
 }

\makeatletter
\def\CTEX@section@format{\Large\bfseries}
\makeatother

\makeatletter
\newenvironment{tablehere}
 {\def\@captype{table}}
 {}

\newenvironment{figurehere}
 {\def\@captype{figure}}
 {}
\makeatother
%
%\newcommand\BackgroundPic{%
%\put(0, 0){%
%\parbox[b][\paperheight]{\paperwidth}{%
%\vfill
%\centering
%\includegraphics[width=\paperwidth, height=\paperheight, %
%%keepaspectratio]{images/background.jpg}%
%]{images/background.jpg}%
%\vfill
%}}}


\begin{document}
%\AddToShipoutPicture{\BackgroundPic}
\maketitle


\begin{abstract}
Loopring es un protocolo abierto que permite construir plataformas de intercambio descentralizado. Loopring opera como un conjunto de contratos inteligentes p\'ublicos, responsables del comercio y de la liquidaci\'on, junto a un grupo de agentes fuera-de-cadena (off-chain) que agregan y comunican pedidos. El protocolo es gratuito, extensible, y sirve como un elemento de soporte estandarizado para la construcci\'on de aplicaciones descentralizadas (dApps), que incorporan la funcionalidad de intercambios. Sus est\'andares interoperables facilitan un comercio an\'onimo que no requiere de confianza (trustless). Una mejora importante, con respecto a los protocolos descentralizados de plataformas de intercambio actuales, es la habilidad para hacer que los pedidos sean mezclados y emparejados con otros pedidos diferentes, obviando las restricciones de formar pares de intercambio entre dos-tokens y mejorando dr\'asticamente la liquidez. Adem\'as, Loopring emplea una soluci\'on \'unica y robusta para la prevenci\'on del front-running: el intento injusto de enviar transacciones a un bloque m\'as r\'apido que el del proveedor de la soluci\'on original. Loopring es agn\'ostica en relaci\'on con las cadenas de bloques (blockchains), y puede implantarse en cualquier blockchain con funcionalidad de contrato inteligente. Hasta el momento de escribir este documento, Loopring es operable en Ethereum \cite{buterin2017ethereum} \cite{wood2014ethereum} y Qtum \cite{dai2017smart} con NEO \cite{atterlonn2018distributed} bajo construcci\'on.
\end{abstract}

\begin{multicols}{2}
\section{Introducci\'on\label{sec:introduction}}
Con la proliferaci\'on de activos basados en la blockchain, la necesidad de intercambiar estos activos entre las contrapartes se ha incrementado considerablemente. A medida que miles de nuevos tokens son introducidos - incluyendo los activos tradicionales de tokenizaci\'on - est\'a necesidad se magnifica. Ya sea intercambiando tokens, por motivaciones comerciales especulativas, o convirti\'endolos para acceder a la red v\'ia sus tokens de utilidad nativa, la habilidad de intercambiar un activo criptogr\'afico es fundamental para un ecosistema amplio. De hecho, hay una energ\'ia potencial en los activos \cite{desotocapital}; liberar esta energ\'ia - desbloqueando el capital - no solo requiere determinar la propiedad, lo que las blockchains han permitido inmutablemente, sino tambi\'en la habilidad de transferir y transformar libremente estos activos.


Como tal, el intercambio de tokens (valor) que no requiere una relaci\'on de confianza, es un caso convincente de uso para la tecnolog\'ia de blockchain. Sin embargo, hasta ahora, los entusiastas de la criptograf\'ia se han conformado, en gran medida, por comerciar tokens en las tradicionales plataformas de intercambio (exchanges, en ingl\'es) centralizado. El protocolo Loopring es necesario porque, al igual que Bitcoin \cite{nakamoto2008bitcoin}, enfatiza diligentemente que, con respecto al dinero electr\'onico entre-pares (peer-to-peer), \enquote{los mayores beneficios se pierden si un tercero confiable a\'un requiere prevenir el gasto-doble}; de la misma manera, los principales beneficios de los recursos descentralizados se pierden, si tienen que pasar por las plataformas de intercambio centralizado, cerrado, que requieren de confianza.

El intercambio descentralizado de tokens en las plataformas de intercambio centralizado no tiene sentido desde una perspectiva filos\'ofica, porque es imposible mantener los valores propugnados por estos proyectos descentralizados. Tambi\'en existen numerosos riesgos pr\'acticos y limitaciones en el uso de las plataformas de intercambio centralizado, que se describir\'an m\'as adelante. Las Plataformas de Intercambio Descentralizado (DEX) \cite{schuh2015bitshares} \cite{bancor} \cite{kyber} han tratado de abordar estos problemas y, en muchos casos, han logrado mitigar los riesgos de seguridad mediante el uso de las blockchains para negociaciones directas. Sin embargo, dado que la capacidad de DEX se convierte en una infraestructura crucial para la nueva econom\'ia, existe un margen considerable para la mejora del rendimiento. Loopring tiene como objetivo proporcionar herramientas modulares para dicha infraestructura, con su protocolo abierto de Aplicaciones Descentralizadas (dApp) agn\'osticas.


\section{Panorama Actual de las Plataformas de Intercambio\label{sec:current_exchange_landscape}}

\subsection{Insuficiencias de las Plataformas de Intercambio Centralizado}
Los tres principales riesgos, asociados con las plataformas de intercambio centralizado, son: 1) la falta de seguridad, 2) la falta de transparencia, y 3) la falta de liquidez.

\textbf{La falta de seguridad} surge del hecho en que los usuarios suelen ceder sus llaves privadas (es decir, fondos) a una entidad centralizada. Esto expone a los usuarios, a la posibilidad de que las plataformas de intercambio centralizado sean presa de los hackers maliciosos. Los riesgos de seguridad y los ciberataques sufridos por todas las plataformas de intercambio son bien conocidos \cite{coincheckhack}  \cite{mcmillan2014inside}, aunque a menudo se aceptan como \enquote{table stakes} (\enquote{mesa de riesgos}), en el comercio del intercambio de los tokens. Al tener bajo su custodia a millones de d\'olares en fondos de los usuarios, en sus servidores, las plataformas de intercambio centralizado siguen siendo atractivas para el ataque de los hackers. Adem\'as, los desarrolladores de plataformas de intercambio tambi\'en pueden cometer errores accidentales honestos con los fondos de los usuarios. Los usuarios, simplemente, no tienen el control de sus tokens cuando los depositan en una plataforma de intercambio centralizado.

\textbf{La falta de transparencia} expone a los usuarios al riesgo de que las deshonestas plataformas de intercambio act\'uen injustamente. La distinci\'on aqu\'i se basa en las intenciones maliciosas de los operadores de las plataformas de intercambio, ya que los usuarios no est\'an realmente comerciando sus propios activos en estas plataformas de intercambio centralizado, sino m\'as bien, hacen un reconocimiento de deuda (IOU). Cuando los tokens son enviados a la cartera de la plataforma de intercambio, este los pone bajo su custodia y ofrece un reconocimiento de la deuda en su lugar. Es as\'i, como todos los intercambios se realizan entre los pagar\'es (IOU) de los usuarios. Para retirarlos, los usuarios canjean sus pagar\'es con la plataforma de intercambio, y reciben sus tokens en la direcci\'on de su cartera externa. En este proceso hay una falta de transparencia, y la plataforma de intercambio puede cerrar, congelar su cuenta, declararse en quiebra, etc. Tambi\'en existe la posibilidad de que utilicen los activos de los usuarios para otros fines mientras los mantienen bajo su custodia, como prestarlos a terceros. La falta de transparencia puede tener un costo para los usuarios, a\'un sin perder el total de sus fondos; por ejemplo, puede generarse tasas de cambio m\'as altas, retrasos en los momentos de mayor demanda, riesgos regulatorios y pedidos siendo \enquote{front-ran}.


\textbf{Falta de Liquidez} Desde el punto de vista de los operadores de las plataformas de intercambio, la liquidez fragmentada impide la entrada de nuevas plataformas de intercambio, debido a la presencia de dos escenarios de \enquote{el-ganador-se-lleva-todo}. En el primero, la plataforma de intercambio con el mayor n\'umero de pares de divisas gana, porque a los usuarios les resulta m\'as ventajoso realizar todos sus intercambios en una sola plataforma de intercambio. En el segundo, la plataforma de intercambio con el mayor registro de pedidos gana, debido al diferencial favorable de la bid-ask (oferta-demanda) para cada uno de los pares del intercambio. Esto desalienta la competencia de las personas nuevas en el mercado de intercambios, porque les resulta dif\'icil acumular la liquidez inicial necesaria. Como resultado, muchas plataformas de intercambio controlan una cuota alta del mercado, a pesar de las quejas de los usuarios y los incidentes sustanciales de ciberataques realizados por los hackers inform\'aticos. Es importante mencionar, que mientras las plataformas de intercambio centralizado m\'as compren y ganen las cuotas del mercado, m\'as expuestos est\'an a los ataques de los hackers.

Desde la perspectiva de los usuarios, la fragmentaci\'on de la liquidez reduce significativamente la Experiencia de Usuario. En una plataforma de intercambio centralizado, los usuarios solo pueden intercambiar dentro de los grupos de liquidez de su misma plataforma de intercambio, contra sus propios libros de pedidos y entre los pares de tokens que soporta. Para intercambiar el token \verb|A| por el token  \verb|B|, los usuarios deben ir a una plataforma de intercambio que soporte ambos tokens o registrarse en diferentes plataformas de intercambio, divulgando as\'i su informaci\'on personal. Los usuarios a menudo necesitan realizar transacciones preliminares o intermedias, generalmente mediante BTC o ETH, lo que lleva a pagar las brechas entre la oferta y la demanda en el proceso. Finalmente, los libros de pedidos pueden no ser lo suficientemente grandes, como para completar la transacci\'on sin una desaceleraci\'on importante. Incluso si la plataforma de intercambio afirma que puede manejar grandes vol\'umenes, no hay garant\'ia de que el manejo de este volumen y liquidez no sean falsos \cite{fakevolume}.


Los resultados son silos de liquidez desconectados y un ecosistema fragmentado que se asemeja al antiguo sistema financiero, que tiene un volumen significante de transacciones centralizadas en unas pocas plataformas de intercambio. Las promesas de la liquidez global de las blockchains no tienen ning\'un m\'erito o valor dentro de las plataformas de intercambio centralizado.

\subsection{Insuficiencia de las Plataformas de Intercambio Descentralizado}
Las plataformas de intercambio descentralizado se diferencian de las plataformas de intercambio centralizado, en parte porque los usuarios mantienen el control de sus llaves-privadas (activos), mediante la ejecuci\'on directa de intercambios en la blockchain subyacente. Aprovechando la \textit{tecnolog\'ia que no requiere de confianza} de las criptomonedas, ellas mismas han mitigado con \'exito muchos de los riesgos, antes mencionados, en torno a la seguridad. Sin embargo, los problemas persisten en relaci\'on con el rendimiento y las limitaciones estructurales.

La liquidez a menudo sigue siendo un problema, ya que los usuarios deben buscar contrapartes, a trav\'es de grupos de liquidez y est\'andares dispares. Los efectos de la liquidez fragmentada est\'an presentes si los DEXs o dApps no utilizan est\'andares consistentes para interoperar, y si los pedidos no son compartidos o propagados dentro de una red amplia. La liquidez de los libros de pedidos limitados y, espec\'ificamente su resiliencia - cu\'an r\'apido los pedidos de limite ejecutados son reemplazados por pedidos nuevos - puede afectar significativamente las estrategias optimas del intercambio \cite{limitorderliquidity}. La ausencia de estos est\'andares ha resultado no s\'olo en la reducci\'on de la liquidez, sino tambi\'en en la exposici\'on a una variedad de contratos inteligentes patentados potencialmente inseguros.


Adem\'as, dado que los intercambios se realizan dentro de la cadena, los DEXs heredan las limitaciones de la blockchain subyacente; nominalmente: escalabilidad, retrasos en la ejecuci\'on (proceso de miner\'ia) y modificaciones costosas en los pedidos. Por lo tanto, el registro de pedidos de la blockchain no es particularmente bien escalado, ya que la ejecuci\'on del c\'odigo en la blockchain genera un costo (gas), lo que hace que las cancelaciones de pedidos m\'ultiples sean excesivamente caras.

Finalmente, dado que los registros de los pedidos de la blockchain son p\'ublicos, la transacci\'on para hacer un pedido es visible para los mineros de criptomonedas, mientras se espera que esta sea extra\'ida en el bloque siguiente y colocada en el libro de pedidos. Este retraso expone a los usuarios al riesgo de una \enquote{ejecuci\'on anticipada} (front-run) y tambi\'en al riesgo de que el precio o la ejecuci\'on sean cambiados en su contra.

\subsection{Soluciones H\'ibridas}
Por las razones mencionadas anteriormente, las plataformas de intercambio \'unicamente basados en la blockchain tienen limitaciones que las hacen no competitivas con las plataformas de intercambio centralizado. Existe un balance entre la falta de confianza caracter\'istica de la on-chain (dentro-de-cadena) y la velocidad y flexibilidad de los pedidos de las plataformas de intercambio centralizado. Protocolos como los de Loopring y 0x \cite{warren20170x} proponen una soluci\'on de liquidaci\'on de transacciones dentro-de-cadena, con una gesti\'on de pedidos fuera-de-cadena. Estas soluciones giran en torno a los contratos inteligentes abiertos; pero superan las limitaciones de escalabilidad mediante la ejecuci\'on de muchas funciones fuera de la cadena, dejando a los nodos flexibilidad para completar diferentes roles cr\'iticos para la red. Sin embargo, las desventajas tambi\'en permanecen para los modelos h\'ibridos \cite{costofdecent}. El protocolo Loopring propone diferencias significativas, en nuestro enfoque para una soluci\'on h\'ibrida a trav\'es del desarrollo del presente art\'iculo.

\section{El protocolo Loopring\label{sec:loopring_protocol}}
Loopring no es una Plataforma de Intercambio Descentralizado (DEX), sino un protocolo modular para construir DEXs en m\'ultiples blockchains. Hemos desmontado partes de los componentes principales de una plataforma de intercambio tradicional y ofrecido un conjunto de contratos inteligentes p\'ublicos y agentes descentralizados en su lugar. Los roles en la red incluyen carteras, rel\'es (relays, en ingl\'es), blockchains de consorcio para compartir liquidez, buscadores de registros de pedidos, Mineros-de-Anillo (Ring-Miners, en ingl\'es) y servicios de tokenizaci\'on de activos. Antes de definir cada uno de estos elementos, primero debemos entender cu\'ales son los pedidos en Loopring.

\subsection{Anillo de Pedidos \label{sec:order_ring}}
Los pedidos en Loopring se expresan en lo que llamamos un Modelo de Pedidos Unidireccional (UDOM, por sus siglas en ingl\'es)\cite{coinport2014udom}. UDOM formula pedidos como solicitudes de intercambio de token, \verb|cantidadS|/\verb|cantidadB|, (cantidad a vender/comprar) en lugar de los pedidos de oferta y demanda. Dado que cada pedido es solo un tipo de cambio entre dos tokens, una caracter\'istica importante del protocolo es la mezcla y coincidencia entre varios pedidos en un intercambio circular. Utilizando hasta 16 pedidos simult\'aneamente, en lugar de un solo par de intercambio, se genera un aumento dr\'astico en la liquidez y un potencial para la mejora del precio.


\begin{center}
\begin{figurehere}
\centering
\tikzstyle{block} = [draw, fill=blue!20, rectangle, 
    minimum height=3em, minimum width=6em]
\tikzstyle{sum} = [draw, fill=blue!20, circle, node distance=1cm]
\tikzstyle{input} = [coordinate]
\tikzstyle{output} = [coordinate]
\tikzstyle{pinstyle} = [pin edge={to-,thin,black}]

\begin{tikzpicture}[
    auto, 
    node distance=2cm,
    >=latex',
    font=\bfseries\footnotesize\sffamily,
    order/.style={
		scale=0.7,
		rectangle,
		rounded corners,
		draw=black, 
		text centered,
%		text width=5cm,
		minimum height=12mm,
		fill=white
	},
	label/.style={
		scale=0.7
	}
  ]
    % We start by placing the blocks

  \node [order] (order2) 
 {%
 \begin{tabular}{l}
  \textbf{PEDIDO\#2}\\
  \textbf{propietario: Y}\\
  \textbf{cantidadS: 9B}\\
  \textbf{cantidadB: 12C}
 \end{tabular}
 };
 
  \node [order, below of=order2, xshift=-3.5cm] (order1) 
 {%
 \begin{tabular}{l}
  \textbf{PEDIDO\#1}\\
  \textbf{propietario: X}\\
  \textbf{cantidadS: 10000A}\\
  \textbf{cantidadB: 2B}
 \end{tabular}
 };
 
 
  \node [order, below of=order2, xshift=3.5cm] (order3) 
 {%
 \begin{tabular}{l}
  \textbf{PEDIDO\#3}\\
  \textbf{propietario: Z}\\
  \textbf{cantidadS: 100C}\\
  \textbf{cantidadB: 160A}
 \end{tabular}
 };
 
 \draw [draw,->] (order1) -- node [label] {\textbf{7898A}} (order3);
 \draw [draw,->] (order2) -| node [label, xshift=-1.8cm] {\textbf{8B}} (order1);
 \draw [draw,->] (order3) |- node [label, xshift=1cm, yshift=0.24cm] {\textbf{98C}} (order2);

\end{tikzpicture}

\caption{Un anillo de pedidos de 3 \'ordenes}
\label{fig:ring}
\end{figurehere}
\end{center}


La Figura de arriba muestra un anillo de 3 pedidos. El token de cada pedido a vender (\verb|tokenS|) es el token de otro pedido a comprar (\verb|tokenB|). Esto crea un bucle que permite que cada pedido intercambie sus tokens deseados, sin necesitar un pedido opuesto a su par. Los pares de pedidos de intercambios tradicionales pueden, por supuesto, seguir ejecut\'andose, en lo que es esencialmente un caso especial de un anillo de pedidos.


\begin{definition}[anillo-de-pedidos] Deje $C_{0}$, $C_{1}$, $\cdots$, $C_{n-1}$ ser $n$ diferentes tokens, $O_{0\rightarrow 1}$, $\cdots$, $O_{i\rightarrow i\oplus 1}$, $\cdots$, $O_{n-1 \rightarrow 0}$ ser $n$ pedidos. Estos pedidos pueden formar un anillo-de-pedidos para intercambiar:
$$O_{0\rightarrow 1} \rightarrow \cdots \rightarrow O_{i\rightarrow i\oplus 1} \rightarrow \cdots \rightarrow O_{n-1\rightarrow 0} \text{, }$$
donde $n$ es la longitud del anillo de pedidos, y $i\oplus 1 \equiv i+1 \mod n$.
\end{definition}

Un anillo de pedidos es v\'alido, cuando todas las transacciones que lo componen pueden realizar a un tipo de cambio igual o superior a la tasa original especificada por el usuario. Para verificar la validez del anillo de pedidos, los contratos inteligentes del protocolo Loopring deben recibir los anillos de pedidos de los mineros, y verificar que el producto de las tasas de cambio originales de todos los pedidos sea mayor o igual a 1.

Supongamos que Alice y Bob quieren intercambiar sus tokens \verb|A| y \verb|B|. Alice tiene 15 token \verb|A| y quiere 4 token \verb|B|; Bob tiene 10 token \verb|B| y quiere 30 tokens \verb|A| a cambio.



?`Qui\'en est\'a comprando y qui\'en est\'a vendiendo? Esto depende solo en los activos que necesitamos fijar para dar las cotizaciones de precios. Si token \verb|A| es la referencia, entonces Alice est\'a comprando token \verb|B| por el precio de ${15 \over 4} = 3.75$\verb|A|, mientras que Bob vende 10 token \verb|B| por el precio de ${30 \over 10} = 3.00$\verb|A|. En el caso de la fijaci\'on de token \verb|B| como referencia, decimos que Alice est\'a vendiendo 15 token \verb|A| por el precio de ${4 \over 15} = 0.26666667$\verb|B| y Bob est\'a comprando 10 token \verb|A| por el precio de ${10 \over 30} = 0.33333334 $\verb|B|. Por lo tanto, qui\'en es el comprador o el vendedor es arbitrario.


En la primera situaci\'on, Alice est\'a dispuesta a pagar un precio m\'as alto ($3.75$\verb|A|) que el precio por el que Bob est\'a vendiendo sus token ($3.00$\verb|A|), mientras que en la segunda situaci\'on Bob est\'a dispuesto a pagar un precio m\'as alto ($ 0.33333334 $\verb|B|) que el precio por el que Alice est\'a vendiendo sus tokens ($ 0.26666667 $\verb|B|). Est\'a claro que un intercambio es posible siempre que el comprador est\'e dispuesto a pagar un precio igual o superior al precio del vendedor.

\begin{equation}
{{15\over 4} \over {30\over 10}} = {{10\over 30} \over {4\over 15}}={15 \over 4} \cdot {10 \over 30} = 1.25 > 1
\end{equation}


Por lo tanto, para que un conjunto de $n$ pedidos pueda llenarse y ejecutarse, total o parcialmente, necesitamos saber si el producto entre cada una de las tasas de cambio de los pedidos de compra resulta en un n\'umero mayor o igual a 1. Si es as\'i, todas los pedidos $n$ pueden ser parcialmente, o totalmente ejecutados \cite {supersymmetry}.


Si presentamos a una tercera contraparte, Charlie, as\'i que si Alice quiere dar $x_1$ token \verb|A| y recibe $y_1$ token \verb|B|, Bob quiere dar $x_2$ token \verb|B| y recibe $y_2$ token \verb|C|, y Charlie quiere dar $x_3$ token \verb|C| y recibir $y_3$ token \verb|A|. Los tokens necesarios est\'an presentes, y el intercambio es posible si:

\begin{equation}
{{x_1 \cdot x_2 \cdot x_3 \over y_1 \cdot y_2 \cdot y_3} \geq 1}
\end{equation}

Ver la secci\'on \ref{anatomy} para m\'as detalles referentes a los pedidos de Loopring.

\section{Participantes del Ecosistema\label{sec:ecosystem}}
Los siguientes participantes del ecosistema, proporcionan conjuntamente todas las funcionalidades que una plataforma de intercambio centralizado ofrece.
\begin{itemize}


\item \textbf{Carteras}: Un servicio o interfaz de la cartera permite a los usuarios acceder a sus tokens y enviar pedidos a la red de Loopring. Las carteras son incentivadas a producir pedidos, al compartir pagos dentro del anillo de pedidos (ver secci\'on \ref{sec:token}). Con la creencia de que el futuro de los intercambios se llevara a cabo de manera segura dentro de las carteras de los usuarios, la conexi\'on de estos fondos de liquidez a trav\'es de nuestro protocolo es primordial.

\item \textbf {Blockchain de Consorcio de Intercambio de Liquidez /Malla-Rel\'e}: Una red de malla de rel\'e (relay mesh network, en ingl\'es) para el intercambio de pedidos \& liquidez. Cuando los nodos ejecutan el software de rel\'e de Loopring, ellos pueden unirse a una red existente y compartir liquidez con otros rel\'es, a trav\'es de una blockchain de consorcio. La blockchain de este tipo, que estamos construyendo como primera implementaci\'on, tiene un tiempo casi real de intercambio de pedidos (bloques de 1-2 segundos), y reduce el historial antiguo para permitir una descarga m\'as r\'apida a los nuevos nodos. Notablemente, los rel\'es no necesitan unirse a este consorcio; pueden actuar solos y no compartir liquidez con otros, o pueden comenzar y administrar su propia red de intercambios de liquidez.


\item \textbf{Rel\'es / Mineros-de-anillos}: Los rel\'es son nodos que reciben pedidos de las carteras o de la malla de rel\'es; mantienen un registro p\'ublico de pedidos y uno de intercambios, y opcionalmente emiten pedidos a otros rel\'es (a trav\'es de cualquier medio fuera de la cadena arbitrario) y/o nodos de malla de rel\'e. La miner\'ia de anillo es una caracter\'istica -- no un requisito -- de los rel\'es. Es computacionalmente intensa y se realiza completamente fuera de la cadena (off-chain). Llamamos a los rel\'es con la funci\'on activada de miner\'ia de anillo \enquote{Ring-Miners}, que produce anillos de pedidos al unir pedidos dispares. Los rel\'es son libres de elegir (1) c\'omo comunicarse entre ellos, (2) c\'omo construir sus libros pedidos, y (3) c\'omo extraer anillos de pedidos (algoritmos de miner\'ia).


\item \textbf{Contratos inteligentes del protocolo Loopring (LPSC)}: Un conjunto de contratos inteligentes p\'ublicos y gratuitos, que verifican los anillos de pedidos recibidos de los mineros de anillo (ring-miners), transfieren los tokens a nombre de los usuarios de manera segura, incentivan las operaciones de las carteras y de los mineros con comisiones, y crean eventos. Los rel\'es/navegadores de pedidos escuchan estos eventos, para mantener actualizados a sus libros de pedidos y su historial comercial. Ver ap\'endice para m\'as detalles. %\ref{app:protocol_ethereum}% 

\item \textbf{Servicios de tokenizaci\'on de activos (ATS)}: Referente a un puente entre activos, que no pueden intercambiarse directamente en Loopring. Estos servicios son administrados centralmente por empresas y organizaciones acreditadas. Los usuarios depositan activos (activos reales, fiat o tokens de otras blockchains) y obtienen tokens que pueden ser redimidos para sus dep\'ositos en el futuro. Looping no es un protocolo de intercambio de cadena-cruzada (hasta que se encuentre una soluci\'on adecuada), pero los servicios de tokenizaci\'on de activos (ATS) permiten el intercambio entre tokens ERC20 \cite{ERC20} y recursos f\'isicos, as\'i como con los recursos presentes en otras blockchains.

\end{itemize}


\section{Proceso de Intercambio\label{sec:process}}
\begin{enumerate} 

\item \textbf{Autorizaci\'on del protocolo}: En la Figura \ref{fig:process}, el usuario \verb|Y|, quiere intercambiar tokens y autoriza a los LPSC a administrar una \verb|cantidadS| del token \verb|B| que el usuario quiere vender. Esta operaci\'on no bloquea los tokens del usuario, quien puede transferirlos libremente mientras se procesa el pedido.

\item \textbf{Creaci\'on del pedido}: El tipo de cambio actual y el registro de pedidos del token \verb|B| vs el token \verb|C|, son proporcionados por los rel\'es o por otros agentes conectados a la red, como los buscadores de pedidos. El usuario \verb|Y| hace un pedido (pedido l\'imite), especificando la \verb|cantidadS| y \verb|cantidadB| y otros par\'ametros a trav\'es de cualquier interface de cartera integrada. Cierta cantidad de LRx se puede agregar al pedido, como una comisi\'on para los mineros de anillo (ring-miners); cu\'an mayor sea la comisi\'on de LRx, mayor ser\'a la probabilidad de que un minero procese r\'apidamente el pedido. El hash del pedido se firma con la llave privada del usuario \verb|Y|.

\item \textbf{Transmisi\'on de pedidos}:La cartera env\'ia el pedido y su firma a uno o m\'as rel\'es. Luego los rel\'es actualizan su libro de pedidos p\'ublico. El protocolo no requiere que los libros de pedidos sean construidos de una manera espec\'ifica, como por ejemplo con la regla de \enquote{el primero en llegar es el primero en ser atendido}. En cambio, los rel\'es tienen el poder de tomar sus propias decisiones de dise\~no, mediante la construcci\'on de sus libros de pedidos.

\item \textbf{Distribuci\'on de liquidez}: Los rel\'es transmiten un pedido a los otros rel\'es, a trav\'es del m\'etodo que consideren m\'as adecuado. De nuevo, hay flexibilidad sobre c\'omo los nodos interact\'uan. Para facilitar el logro de un cierto nivel de conectividad, se ha implementado un mecanismo incorporado para compartir la liquidez entre las mallas de rel\'es (relay-mesh), utilizando un consorcio de blockchains. Como se mencion\'o en la secci\'on anterior, esta malla de rel\'e est\'a optimizada para la velocidad y la inclusi\'on.

\begin{center}
\begin{figurehere}
\centering
\tikzstyle{block} = [draw, fill=blue!20, rectangle, 
    minimum height=3em, minimum width=6em]
\tikzstyle{sum} = [draw, fill=blue!20, circle, node distance=1cm]
\tikzstyle{input} = [coordinate]
\tikzstyle{output} = [coordinate]
\tikzstyle{pinstyle} = [pin edge={to-,thin,black}]

\begin{tikzpicture}[
    auto, 
    scale=0.7,
    node distance=2.2cm, %CHANGED ring-mining Order1 height%
    >=latex',
    font=\bfseries\footnotesize\sffamily,
    order/.style={
		rectangle,
		scale=0.7,
		rounded corners,
		draw=black, 
		text centered,
%		text width=5cm,
		minimum height=12mm,
		minimum width=30mm,
		fill=white
	},
	role/.style={
		circle,
		scale=0.7,
		draw=black, 
		text centered,
%		text width=5cm,
		minimum height=12mm,
		minimum width=12mm,
		fill=white
	},
	steps/.style={
		circle,
		scale=0.7,
		draw=black, 
		text centered,
%		text width=5cm,
%		minimum height=12mm,
%		minimum width=12mm,
		fill=black,
		text=white
	},
	account/.style={
		circle,
		scale=0.7,
		draw=black, 
		text centered,
%		text width=5cm,
		minimum height=16mm,
		minimum width=16mm,
		fill=white
	},
	label/.style={
	  scale=0.7
    }
  ]

 
 \node [role] (user1)  {usuario X};
 \node [role, below of=user1] (user2)  {usuario Y};
 \node [role, below of=user2] (user3)  {usuario Z};
 \node [role, below of=user3, fill=gray!20] (relay1)  {rel\'e M};
 \node [role, below of=relay1, fill=gray!20] (relay2)  {rel\'e N};

 
 \node [order, left of=user1, xshift=-1cm] (order1) 
 {%
 \begin{tabular}{l}
  \textbf{PEDIDO 1}\\
  \textbf{propietario: X}\\
  \textbf{cantidadS: 10000 A}\\
  \textbf{cantidadB: 2 B}
 \end{tabular}
 };
 
 \draw [draw, ->]  (user1) -- (order1) [label]{};
 \draw [bend right,->] (order1) to node [auto, scale=0.7] {} (relay1);
 \draw [bend right,->] (order1) to node [auto, scale=0.7] {} (relay2);
% \draw [draw, ->]  (order1) |- (relay1) [label]{};
% \draw [draw, ->]  (order1) |- (relay2) [label]{};
 
 \node [order,left of=user2, xshift=-1.5cm] (order2) 
 {%
 \begin{tabular}{l}
  \textbf{PEDIDO 2}\\
  \textbf{propietario: Y}\\
  \textbf{cantidadS: 9  B}\\
  \textbf{cantidadB: 12 C}
 \end{tabular}
 };
 \draw [draw, ->]  (user2) -- (order2) [label]{};
 \draw [bend right,->] (order2) to node [auto, scale=0.7] {} (relay1);
 \draw [bend right,->] (order2) to node [auto, scale=0.7] {} (relay2);
% \draw [draw, ->]  (order2) |- (relay1) [label]{};
% \draw [draw, ->]  (order2) |- (relay2) [label]{};
% 
\node [order, left of=user3, xshift=-2cm] (order3) 
 {%
 \begin{tabular}{l}
  \textbf{PEDIDO 3}\\
  \textbf{propietario: Z}\\
  \textbf{cantidadS: 100 C}\\
  \textbf{cantidadB: 160 A}
 \end{tabular}
 };
 \draw [draw, ->]  (user3) -- (order3) [label]{};
 \draw [bend right,->] (order3) to node [auto, scale=0.7] {} (relay1);
 \draw [bend right,->] (order3) to node [auto, scale=0.7] {} (relay2);
% \draw [draw, ->]  (order3) |- (relay1) [label]{};
% \draw [draw, ->]  (order3) |- (relay2) [label]{};
 
% // The Ring
\node [order, 
yshift=-1.5cm,
xshift=-2.75cm,
below of=relay2,
fill=gray!10,
minimum width=4.2cm,
minimum height=5.4cm] (ring) {}; %changed size%


\node [order, dashed, below of=relay2,yshift=-0.2cm,xshift=-2.5cm] (order11) 
 {%
 \begin{tabular}{l}
  \textbf{PEDIDO 1}\\
  \textbf{propietario: X}\\
  \textbf{cantidadS: 10000 A}\\
  \textbf{cantidadB: 2 B}
 \end{tabular}
 };
 \node [order, dashed,below of=order11,xshift=-0.25cm,yshift=0.7cm] (order21) 
 {%
 \begin{tabular}{l}
  \textbf{PEDIDO 2}\\
  \textbf{propietario: Y}\\
  \textbf{cantidadS: 9  B}\\
  \textbf{cantidadB: 12 C}
 \end{tabular}
 };
\node [order, dashed,below of=order21,xshift=-0.25cm,yshift=0.7cm] (order31) 
 {%
 \begin{tabular}{l}
  \textbf{PEDIDO 3}\\
  \textbf{propietario: Z}\\
  \textbf{cantidadS: 100 C}\\
  \textbf{cantidadB: 160 A}
 \end{tabular}
 };
 
 % // The blockchain
\node [
rectangle,
fill=gray!20, 
right of=user1,
yshift=-4.5cm,
xshift=0.1cm,
scale=0.7,
minimum width=3.2cm,
minimum height=15.6cm] (blockchain) {\parbox[b][15cm]{1.3cm}{blockchain}};
% blockchain accounts
  \node [account, right of=user1, xshift=1cm] (account1)  {cuentaX};
  \node [account, right of=user2, xshift=1cm] (account2)  {cuentaY};
  \node [account, right of=user3, xshift=1cm] (account3)  {cuentaZ};
  \node [account, right of=relay1, xshift=1cm] (account4)  {cuentaM};
  \node [account, right of=relay2, xshift=1cm] (account5)  {cuentaN};
  \node [account, double, below of=account5, yshift=-1.5cm] (psc)  {LPSC};
  
 \draw [draw, ->]  (user1) -- (account1) [label]{};
 \draw [draw, ->]  (user2) -- (account2) [label]{};
 \draw [draw, ->]  (user3) -- (account3) [label]{};
% \draw [draw, ->]  (relay1) -- (account4) [label]{};
% \draw [draw, ->]  (relay2) -- (account5) [label]{};
 \draw [draw, double, thick]  (relay1) to node [auto, scale=0.7] {compartir liquidez}  (relay2) [label]{};
% \draw [draw, ->]  (relay1) -- (ring) [label]{};
 \draw [draw, ->]  (relay2) to node [auto, scale=0.7, xshift=-2.3cm, yshift=0.3cm] {miner\'ia de anillo}  (ring) [label]{};
 \draw [draw, ->]  (ring) to node [auto, scale=0.7] {enviarAnillo} (psc) [label]{};
 
 \draw [bend left,->] (account1) to node [auto, scale=0.7] {\textbf{7898 A}} (account3);
 \draw [bend left,->] (account2) to node [auto, scale=0.7] {\textbf{8 B}} (account1);
 \draw [bend left,->] (account3) to node [auto, scale=0.7] {\textbf{98 C}} (account2);
 
 \draw [bend left,->, dashed] (account1) to node [auto, scale=0.7] {} (account5);
 \draw [bend left,->, dashed] (account2) to node [auto, scale=0.7] {} (account5);
 \draw [bend left,->, dashed] (account3) to node [auto, scale=0.7, xshift=.5cm] {\textbf{comisi\'on}} (account5);
  
  
% \draw [draw,->] (order1) -- node [label] {\textbf{7898 A}} (order3);
% \draw [draw,->] (order2) -| node [label, xshift=-1.8cm] {\textbf{8 B}} (order1);
% \draw [draw,->] (order3) |- node [label, xshift=1cm, yshift=0.24cm] {\textbf{98 C}} (order2);

\node [steps, right of=user2, xshift=-0.6cm] () {1};
\node [steps, left of=user2, xshift=0.8cm] () {2};
\node [steps, left of=relay2, xshift=0.3cm, yshift=1cm] () {3};
\node [steps, left of=relay1, xshift=3.3cm, yshift=-1.6cm] () {4};
\node [steps, below of=relay2, xshift=-0.2cm, yshift=0.4cm] () {5};
\node [steps, right of=account3, xshift=-0.6cm] (step5) {6};

 \draw [bend right, ->]  (psc) to node [auto, scale=0.7, xshift=0.5cm] {liquidaci\'on} (step5) [label]{};
 
\end{tikzpicture}

\caption{Proceso de Intercambio EN Loopring}
\label{fig:process}
\end{figurehere}
\end{center}

\item \textbf{Miner\'ia de anillo (Coincidencia de Pedidos)}:  Los mineros de anillo intentan ejecutar el pedido total o parcialmente a un tipo de tasa de intercambio o mejor, emparej\'andolo con varios otros pedidos. La miner\'ia de anillo es la raz\'on principal por la cual el protocolo puede proporcionar alta liquidez sobre cualquier otro par. Si la velocidad en la que se ejecuta el pedido es mejor que la especificada por el usuario \verb|Y|, el margen se comparte entre todos los pedidos que forman el anillo. Como recompensa, el minero puede elegir entre reclamar una parte del margen (Divisi\'on de Margen, y devolver a los usuarios el LRx), o simplemente quedarse con la comisi\'on en LRx.


\item \textbf{Verificaci\'on \& Transacci\'on}: Los LPSC reciben el anillo de pedidos. Se necesitan varios controles para verificar los datos proporcionados por el minero del anillo, y para determinar si el ciclo de pedidos puede desarrollarse completa o parcialmente (dependiendo de la tasa de ejecuci\'on de los pedidos en el anillo y los tokens en las carteras de los usuarios). Si todos los controles son exitosos, el contrato autom\'aticamente transfiere los tokens a los usuarios y paga a los mineros de anillo y a las carteras al mismo tiempo. Si los LPSC detectan la falta de fondos necesarios para el intercambio en la cartera del usuario \verb|Y|, el pedido ser\'a reducido: un pedido de escala reducida volver\'a autom\'aticamente a su tama\~no original, si se deposita una cantidad suficiente de fondos a la direcci\'on del usuario; lo que lo hace diferente de una cancelaci\'on, que es una operaci\'on manual en un solo sentido, que no se puede revocar.

\end{enumerate}





%
%\end{multicols}
%
%\begin{center}
%\begin{figurehere}
%\includegraphics[height=8cm]{images/en_protocol.png}
%\caption{Loopring Trading Process}
%\label{fig: Loopringrotocol}
%\end{figurehere}
%\end{center}
%
%\begin{multicols}{2}


\section{Flexibilidad operacional\label{sec:business_model}}
Es importante, tener en cuenta que los est\'andares abiertos de Loopring permiten a los participantes operar con gran flexibilidad. Los participantes son libres de implementar nuevos modelos de negocio y generar un valor para los usuarios, ganando comisiones en LRx, en vol\'umenes de negociaci\'on u otras m\'etricas definidas en el proceso (si as\'i lo deciden). El ecosistema es modular y est\'a dise\~nado para fomentar la participaci\'on de una multitud de aplicaciones.

\subsection{Registro de pedidos\label{sec:order_book}}
Los rel\'es pueden dise\~nar sus libros de pedidos en cualquier n\'umero de maneras, para mostrar y hacer coincidir los pedidos de los usuarios. Una primera implementaci\'on de nuestro libro de pedidos sigue un modelo de venta-libre (OTC, en ingl\'es), donde el l\'imite de los pedidos se basa solo en el precio. En otras palabras, la marca del tiempo en los pedidos no tiene un impacto en la formaci\'on del libro de pedidos. Sin embargo, un rel\'e es libre de dise\~nar su propio libro de pedidos, en tal manera de emular y competir con el funcionamiento t\'ipico de las plataformas de intercambio centralizado, donde los pedidos se clasifican por precio, mientras tambi\'en se respeta el momento de creaci\'on del pedido. Si un rel\'e est\'a m\'as inclinado a ofrecer este tipo de libro de pedidos, ellos pueden apropiarse/integrarse con una cartera, y hacer que los pedidos desde ella se env\'ien \'unicamente a un solo rel\'e, quien luego podr\'a emparejar esos pedidos en base al tiempo. Cualquier tipo de configuraci\'on es posible.
Mientras que otros protocolos DEX a veces requieren de Rel\'es para tener recursos - saldos de tokens iniciales, para colocar \enquote{pedidos del comprador} (taker orders, en ingl\'es) - Los Rel\'es de Loopring solo necesitan encontrar pedidos que correspondan a otros, para realizar un intercambio, y pueden hacerlo sin tokens iniciales.



\subsection{Compartir la liquidez\label{sec:liquidity_sharing}}
Los rel\'es son libres de dise\~nar c\'omo compartir la liquidez (pedidos) entre ellos. Nuestro consorcio de blockchain es solo una de las soluciones para lograr este objetivo, y el ecosistema es libre de conectarse y comunicarse como lo desee. Adem\'as de unirse al consorcio de blockchain, pueden construir y administrar lo suyo, creando reglas/incentivos tal como lo deseen. Los rel\'es tambi\'en pueden funcionar en forma aut\'onoma, como se ve en el caso de la implementaci\'on de la cartera que toma en cuenta el tiempo. Ciertamente, existen claras ventajas de comunicarse con otros rel\'es en busca de obtener efectos de la red; sin embargo, los diferentes modelos comerciales podr\'ian merecer su propio dise\~no de intercambio de liquidez y las divisiones de comisiones de muchas maneras.

\section{Especificaci\'on del protocolo\label{sec:protocol}}

\subsection{Anatom\'ia de un Pedido\label{anatomy}}
Un pedido es un paquete de datos que describe la intenci\'on del usuario de intercambiar. Un pedido de Loopring se define utilizando el Modelo de Pedidos UniDireccional, o UDOM por sus siglas en ingl\'es, de la siguiente manera:
\begin{verbatim}
  message Order {
    address protocol;
    address owner;
    address tokenS;
    address tokenB;
    uint256 amountS;
    uint256 amountB;
    unit256 lrcFee
    unit256 validSince; // Seconds since epoch
    unit256 validUntil; // Seconds since epoch
    uint8   marginSplitPercentage;  // [1-100]
    bool    buyNoMoreThanAmountB;
    uint256 walletId;
    // Dual-Authoring address
    address authAddr;
   	// v, r, s son parte de la firma
    uint8   v;       
    bytes32 r;
    bytes32 s;
    // llave-privada de Dual_authoring
    // no utilizada para calcular el hash 
    //del pedido,
    // por lo tanto, NO es firmada.
    string  authKey;          
    uint256 nonce;
  }
\end{verbatim}

Para garantizar el origen del pedido, se firma contra el hash de sus par\'ametros, excluyendo la \verb|authAddr|, con la llave privada del usuario. El par\'ametro de \verb|authAddr| se utiliza para firmar los anillos de pedidos de los que forma parte este pedido, lo que impide el front-running. Por favor refi\'erase a la secci\'on \ref{sec:dual_authoring} para m\'as detalles. La firma est\'a representada por los campos \verb|v|, \verb|r|, y \verb|s|, y se env\'ia con los par\'ametros del pedido a la red. Esto asegura que el pedido permanezca inmutable a lo largo de su toda existencia. Incluso si este pedido nunca cambia, el protocolo a\'un puede calcular su estado actual en funci\'on del balance de su direcci\'on y otras variables.


El Modelo de Pedidos UniDireccional (UDOM) no incluye un precio (que debe ser un n\'umero de coma flotante por naturaleza), sino m\'as bien usa el t\'ermino de \verb|tasa| o $r$, que se expresa en \verb|cantidadS|/\verb|cantidadB|. La tasa no es un n\'umero de coma flotante, sino una expresi\'on que solo se evaluar\'a con otros n\'umeros enteros no firmados cuando se solicite, para mantener todos los resultados intermedios como n\'umeros enteros no firmados y para aumentar la precisi\'on del c\'alculo.


\subsubsection{Importes de compra}
Cuando un minero de anillo consigue coincidir los anillos de pedidos, es posible que una mejor tasa sea ejecutable, lo que permitir\'a a los usuarios obtener 5 veces m\'as del \verb|tokenB| de la \verb|amountB| (cantidadB) que ellos especificaron. Sin embargo, si el par\'ametro de \verb|buyNoMoreThanAmountB| se establece en \verb|True| (verdad), el protocolo garantiza que los usuarios reciban no m\'as de la \verb|amountB| (cantidad B) del \verb|tokenB|. Asi, el par\'ametro de UDOM \verb|buyNoMoreThantokenB| determina cu\'ando se debe considerar que un pedido se ha cumplido por completo. \verb|buyNoMoreThantokenB| aplica un valor m\'aximo a \verb|amountS| o \verb|amountB|, y permite a los usuarios expresar sus intenciones de intercambio de forma m\'as detallada que los pedidos de venta y compra tradicionales.

Por ejemplo: con \verb|amountS| = 10 y \verb|amountB| = 2, la tasa $r$ = 10/2 = 5. Por lo tanto, el usuario est\'a dispuesto a vender 5  \verb|tokenS| por cada \verb|tokenB|. El minero empareja y encuentra una tasa de 4 al usuario, lo que le permite recibir 2.5 \verb|tokenB| en lugar de 2. No obstante, si el usuario solo quiere 2 \verb|tokenB| y establece el par\'ametro \verb|buyNoMoreThanAmountB| a \verb|True|, los LPSC ejecutan la transacci\'on a una tasa de 4, y el usuario vende 4 \verb|tokenS| por cada \verb|tokenB|, con un ahorro efectivo de 2 \verb|tokenS|. Considerar que las comisiones no se consideran aqu\'i.  (Ver la secci\'on \ref{sec:fee_model}).

De hecho, si usamos


\begin{verbatim}
	      Order(amountS,tokenS,
	            amountB,tokenB,
	            buyNoMoreThantokenB)
\end{verbatim}

para representar un pedido en una forma simplificada, para los mercados ETH/USD, en una plataforma de intercambio tradicional, el modelo tradicional de compra-venta puede expresar el primer y el tercer pedido de abajo, pero no los otros dos:

\begin{enumerate}
	\item Vende 10 ETH al precio de 300 USD/ETH. Este pedido tambi\'en puede expresarse como: \verb|Order(10, ETH, 3000, USD, Falso)|.
	\item Vende ETH al precio de 300 USD/ETH para obtener 3000 USD. Este pedido puede expresarse como: \verb|Orden(10, ETH, 3000, USD, Verdadero)|.
	\item Compre 10 ETH al precio de 300 USD/ETH. Este pedido tambi\'en puede expresarse como: \verb|Orden(3000, USD, 10, ETH, Verdadero)|.
	\item Gaste 3000 USD para comprar la mayor cantidad de ETH posible a un precio de 300 USD/ETH. Este pedido puede expresarse como: \verb|Orden(3000, USD, 10, ETH, Falso)|.
\end{enumerate}

\subsection{Verificaci\'on de anillo\label{sec:ring_verification}}
Los contratos inteligentes del protocolo Loopring no calculan la tasa de cambio o las cantidades, pero deben recibir y verificar lo que los mineros de anillo proporcionan para estos valores. Estos c\'alculos son realizados por los mineros de anillo, por dos razones principales: (1) el lenguaje de programaci\'on usado para los contratos inteligentes, como la Solidity  \cite{dannen2017introducing} en Ethereum, no es compatible con el proceso matem\'atico de la coma flotante, especialmente pow (x, 1/n) (calculando la ra\'iz n-\'esima de un n\'umero de coma flotante), y (2) es deseable que dicho c\'alculo se realice fuera de la cadena para reducir las operaciones y el costo del uso de la blockchain.


\subsubsection{Verificaci\'on de anillo secundario/Sub-Ring\label{sec:sub_ring_check}}
Este paso impide a los arbitrajistas la posibilidad de obtener injustamente el margen completo de un anillo de pedidos mediante la implementaci\'on de nuevos pedidos dentro de \'el. B\'asicamente, una vez que un minero identifica un anillo de pedidos v\'alido, podr\'ia caer en la tentaci\'on de agregar m\'as pedidos al mismo anillo de pedidos, de modo que se absorba completamente el margen del usuario (tasa de descuento). Como se muestra en la siguiente Figura \ref{fig:subring}, al calcular cuidadosamente $x1$, $y1$, $x2$ and $y2$ ser\'a posible hacer que el producto de todas las tasas de pedidos sea igual a 1, para cancelar cualquier tipo de cambio.

\begin{center}
\begin{figurehere}
\centering
\tikzstyle{block} = [draw, fill=blue!20, rectangle, 
    minimum height=3em, minimum width=6em]
\tikzstyle{sum} = [draw, fill=blue!20, circle, node distance=1cm]
\tikzstyle{input} = [coordinate]
\tikzstyle{output} = [coordinate]
\tikzstyle{pinstyle} = [pin edge={to-,thin,black}]

\begin{tikzpicture}[
    auto, 
    node distance=2cm,
    >=latex',
    font=\bfseries\footnotesize\sffamily,
    order/.style={
		scale=0.7,
		rectangle,
		rounded corners,
		draw=black, 
		text centered,
%		text width=5cm,
		minimum height=12mm,
		fill=white
	},
	label/.style={
		scale=0.7
	}
  ]
    % We start by placing the blocks

  \node [order] (order2) 
 {%
 \begin{tabular}{l}
  \textbf{PEDIDO 2}\\
  \textbf{propietario: Y}\\
  \textbf{cantidadS: 9B}\\
  \textbf{cantidadB: 12C}
 \end{tabular}
 };
 
  \node [order, below of=order2, xshift=-3.5cm] (order1) 
 {%
 \begin{tabular}{l}
  \textbf{PEDIDO 1}\\
  \textbf{propietario: X}\\
  \textbf{cantidadS: 10000 A}\\
  \textbf{cantidadB: 2 B}
 \end{tabular}
 };
 
 
  \node [order, below of=order2, xshift=3.5cm] (order3) 
 {%
 \begin{tabular}{l}
  \textbf{PEDIDO 3}\\
  \textbf{propietario: Z}\\
  \textbf{cantidadS: 100 C}\\
  \textbf{cantidadB: 160 A}
 \end{tabular}
 };
 
   \node [order, below of=order3, fill=gray!20] (order4) 
 {%
 \begin{tabular}{l}
  \textbf{PEDIDO 4}\\
  \textbf{propietario: M}\\
  \textbf{cantidadS: x1 A}\\
  \textbf{cantidadB: y1 B}
 \end{tabular}
 };
 
 
  \node [order, below of=order1, fill=gray!20] (order5) 
 {%
 \begin{tabular}{l}
  \textbf{PEDIDO 5}\\
  \textbf{propietario: direcci\'onM}\\
  \textbf{cantidadS: x2 C}\\
  \textbf{cantidadB: y2 A}
 \end{tabular}
 };
 
 \draw [draw,->] (order1) -- node [label, xshift=-2cm] {} (order5);
 \draw [draw,->] (order2) -| node [label, xshift=-1.6cm] {} (order1);
 \draw [draw,->] (order3) |- node [label, xshift=1cm] {} (order2);
 \draw [draw,->] (order4) -- node [label, xshift=1.8cm] {} (order3);
 \draw [draw,->] (order5) -- node [label, yshift=0.2cm] {} (order4);
  
\end{tikzpicture}

\caption{Un anillo de pedidos con un sub-anillo}
\label{fig:subring}
\end{figurehere}
\end{center}

Esto es de cero-riesgo, cero-valor a\~nadido a la red, y se considera una conducta injusta por parte del minero de anillos. Para prevenir esto, Loopring requiere que un bucle v\'alido no contenga ning\'un sub-ring. Para verificar esto, los LPSC aseguran que un token no pueda estar en una posici\'on de compra o venta dos veces. En el diagrama anterior, podemos ver que el token \verb|A| es un token de venta dos veces y un token de compra dos veces, lo que no ser\'ia permitido.



\subsubsection{Verificaci\'on de la tasa de ejecuci\'on\label{sec:fill_rate_check}}

El c\'alculo de la tasa de cambio en el anillo de pedidos es realizado por los mineros, por las razones explicadas anteriormente. Son los LPSC los que deben verificar que esta operaci\'on sea correcta. Primero, verifica que la tasa de compra que el minero de anillo puede ejecutar por cada pedido, sea menor o igual a la tasa de compra original impuesta por el usuario. Esto asegura que el usuario obtenga al menos el tipo de cambio que solicit\'o, o algo mejor en la transacci\'on.

Luego de la confirmaci\'on de las tasas de cambio, los LPSC se aseguran de que cada pedido que compone el anillo se beneficie del mismo descuento. Por ejemplo, si la tasa de descuento es $\gamma$ el precio de cada pedido ser\'a


$r_{0\rightarrow 1} \cdot (1-\gamma)$, $r_{1\rightarrow 2} \cdot (1-\gamma)$, $r_{2 \rightarrow 0} \cdot (1-\gamma)$, y satisface: 
\begin{equation}
r_{0\rightarrow 1} \cdot (1-\gamma)\cdot r_{1\rightarrow 2} \cdot (1-\gamma) \cdot r_{2 \rightarrow 0} \cdot (1-\gamma) = 1
\end{equation}
por lo tanto: 
\begin{equation}
\gamma = 1- \frac{1}{\sqrt[3]{r_{0\rightarrow 1} \cdot r_{1\rightarrow 2} \cdot r_{2\rightarrow 0}}}\text{.}
\end{equation}
Si la transacci\'on agrega $n$ pedidos, el \texttt{descuento} es: 
\begin{equation}
\gamma = 1- \frac{1}{\sqrt[n]{\prod_{i=0}^{n-1} r^i}} \text{,}
\end{equation}

donde $r^i$ representa la tasa de rotaci\'on de la $i$-\'esimo pedido. Obviamente, solo cuando el descuento es $\gamma \ge 0$, estos pedidos pueden ser completados; y la tasa de cambio real del pedido $i$-\'esimo ($O^i$) es: $\hat{r^i} = r^i \cdot (1-\gamma)$, $\hat{r^i}\le r^i$.

Recuerde nuestro ejemplo anterior en el que Alice tiene 15 token \verb|A|  y quiere cambiarlos por 4 token \verb|B| y Bob tiene 10 token \verb|B| y quiere 30 token \verb|A|. Si tomamos al token \verb|A| como referencia, entonces Alice esta comprando token \verb|B| por un valor de $\frac{15}{4}$ = 3.75\verb|A|, mientras que Bob est\'a vendiendo token \verb|B| por $\frac{30}{10}$ = 3.00\verb|A|. Para calcular este descuento: $\frac{150}{120}$ = 1.25 por lo tanto $\frac{1}{1.25}$ = 0.8 = $(1 −- \gamma)^2$. As\'i, la tasa de cambio que ejecuta el intercambio equitativo para ambas partes es $\sqrt{0.8}$ $\cdot$ 3.75 $\approx$ 3.3541 token \verb|A| por token \verb|B|.

Bob da 4 tokens \verb|B| y recibe 13.4164 token \verb|A|, m\'as de los 12 tokens que esperaba por sus 4 tokens. Alice recibe los 4 token \verb|B| que esperaba, pero paga solo 13.4164 token \verb|A| a cambio, menos de los 15 que estaba dispuesta a pagar. Tenga en cuenta que una fracci\'on de este margen se utilizar\'a para pagar las comisiones utilizadas para incentivar a los mineros (y carteras). (Ver la secci\'on \ref{sec:fee_model}).


\subsubsection{Seguimiento de finalizaci\'on \& cancelaci\'on}
Un usuario puede cancelar parcial o totalmente un pedido enviando una transacci\'on espec\'ifica a los LPSC, que contiene los detalles del pedido y el importe que se cancelar\'a. Los LPSC toman nota de ello, registran la cantidad que se cancelar\'a y emiten un evento de \verb|OrderCancelled| a la red. Los LPSC mantienen y rastrean las cantidades ejecutadas y eliminadas a trav\'es de un registro que usa el hash del pedido como un identificador. Esta informaci\'on es de acceso p\'ublico y los eventos de la \verb|OrderCancelled| / \verb|OrderFilled| se emiten con cada cambio. El seguimiento o rastreo de estos valores es cr\'itico para los LPSCs durante el paso de la liquidaci\'on de anillos de pedidos. 

Los LPSC tambi\'en permiten la cancelaci\'on de un pedido por cualquier par de cambio mediante el evento de \verb|OrdersCancelled| y cancelaci\'on de todos los pedidos relacionados con una direcci\'on a trav\'es del evento \verb|AllOrdersCancelled|.


\subsubsection{Escala de pedidos\label{sec:order_scaling}}
Los pedidos son escalados y actualizados en base al historial de importes ejecutados y a los importes cancelados, as\'i como el saldo actual de la cuenta del remitente. En base a estas caracter\'isticas, el proceso identifica el pedido con la cantidad m\'as baja que se ejecutar\'a, y utiliza esas caracter\'isticas como referencia para escalar todas las transacciones del anillo de pedidos.

La identificaci\'on del pedido con el valor m\'as bajo puede ayudar a facilitar la estimaci\'on del volumen de ejecuci\'on de cada pedido. Por ejemplo, suponiendo que el pedido $i$--\'enesimo es el que tiene el valor m\'as bajo, el n\'umero de tokens vendidos por cada pedido $\hat{s}$ y el n\'umero de tokens comprados $\hat{b}$  por cada pedido se puede calcular como: 

\[
\begin{split}
&\hat{s}^{i}=\overline{s}_i\text{, } \hat{b}^{i}=\hat{s}^{i}/ \hat{r}^i\text{, }\text{;}\\
&\hat{s}^{i\oplus 1}=\hat{b}^i\text{, } \hat{b}^{i\oplus 1}=\hat{s}^{i\oplus 1}/ \hat{r}^{i\oplus 1}\text{;}\\
&\hat{s}^{i\oplus 2}=\hat{b}^{i\oplus 1}\text{, } \hat{b}^{i\oplus 2}=\hat{s}^{i\oplus 2}/ \hat{r}^{i\oplus 2}\text{;}\\
& ...
%\text{.}
\end{split}
\]

donde $\overline{s}_i$ es el saldo restante despu\'es que los pedidos se han ejecutado parcialmente.

En la fase de implementaci\'on, podemos suponer con seguridad que cualquier pedido del anillo tiene el valor m\'as bajo, y luego iterar a trav\'es del anillo de pedidos, como mucho dos veces, para calcular el volumen de ejecuci\'on de cada pedido.


Ejemplo: si la cantidad m\'as baja que debe ejecutarse representa el 5\% del pedido original, todas las transacciones en el anillo de pedidos se reducir\'an en un 5\%. Una vez que las transacciones se completen, el pedido que se consider\'o que tiene la cantidad m\'as peque\~na restante para ser llenada, tendr\'a que ejecutarse por completo.


\subsection{Liquidaci\'on del anillo\label{sec:settlement}}
Si el anillo de pedidos satisface todos los puntos anteriores, el anillo de pedidos puede ser cerrado para permitir la ejecuci\'on de las transacciones. Esto significa que todos los pedidos $n$ formar\'an un ciclo cerrado, conectados como en la Figura 4:


\begin{center}
\begin{figurehere}
\centering
\begin{tikzpicture}[
circle/.style={
		scale=0.75,
		rounded corners,
		draw=black, 
		text centered,
		}
]

\def \n {6}
\def \m {4}
\def \radius {1.4cm}
\def \margin {12} % margin in angles, depends on the radius

\foreach \s in {1,...,\m}
{
  \node[draw, circle] at ({360/\n * (\s - 1)}:\radius) {$O^\s$};
  \draw[<-, >=latex] ({360/\n * (\s - 1)+\margin}:\radius) 
    arc ({360/\n * (\s - 1)+\margin}:{360/\n * (\s)-\margin}:\radius);
}

\node[draw, circle] at ({360/\n * 4}:\radius) {$O^5$};
  \draw[<-, dashed, >=latex] ({360/\n * 4+\margin}:\radius) 
    arc ({360/\n * 4+\margin}:{360/\n * (5)-\margin}:\radius);
    
\node[draw, circle] at ({360/\n * 5}:\radius) {$O^n$};
  \draw[<-, >=latex] ({360/\n * 5+\margin}:\radius) 
    arc ({360/\n * 5+\margin}:{360/\n * (6)-\margin}:\radius);


\end{tikzpicture}
\caption{Liquidaci\'on del anillo}
\label{fig:settlement}
\end{figurehere}
\end{center}

Para realizar transacciones, los LPSC usan el contrato inteligente \verb|TokenTransferDelegate|. La introducci\'on de ese sistema hace que sea m\'as f\'acil actualizar cualquier contrato inteligente del protocolo; ya que todos los pedidos solo tendr\'an que autorizar a este delegado, en lugar de las diferentes versiones del protocolo. Por cada pedido en el anillo de pedidos, un pago de \verb|tokenS| es hecho al pedido siguiente o anterior, dependiendo de la implementaci\'on. Luego, la comisi\'on del minero se paga de acuerdo con el modelo de comisiones elegido por el propio minero. Finalmente, una vez que se realizan todas las transacciones, se emite el evento \verb|RingMined|.


\subsubsection{Eventos emitidos\label{sec:events}}

El protocolo emite eventos que permiten que los rel\'es y otros participantes involucrados reciban actualizaciones del libro de pedidos, de la manera m\'as eficiente posible. Estos eventos son:

\begin{itemize}
	\item \textbf{OrderCancelled}: Un pedido espec\'ifico ha sido cancelado.
	\item \textbf{OrdersCancelled}: Todos los pedidos de un par de intercambio especifico de una sola direcci\'on, han sido cancelados.
	\item \textbf{AllOrdersCancelled}: Todos los pedidos de una sola direcci\'on han sido cancelados.
	\item \textbf{RingMined}: Un anillo de pedidos se ha establecido con \'exito. Este evento contiene datos relacionados con las transferencias de cada token del anillo-interior.
\end{itemize}

\section{Token LRx\label{sec:token}}
LRx es nuestra forma gen\'erica de nombrar tokens. LRC es el token de Loopring en Ethereum, LRQ en Qtum, LRN en NEO, etc. Se introducir\'an otros tipos de LRx en el futuro, ya que Loopring se extender\'a a otras cadenas de bloques p\'ublicas.


\subsection{Modelo de Comisiones\label{sec:fee_model}} 
Cuando los usuarios crean un pedido, especifican una cantidad de LRx a pagar al minero como comisi\'on, en conjunto con un porcentaje del margen (\verb|marginSplitPercentage|), que el minero puede solicitar. A esto se llama margen dividido. Depende del minero de anillos decidir qu\'e opci\'on escoger\'a entre una comisi\'on o un margen dividido. 
Aqu\'i una representaci\'on del margen dividido:

\begin{center}
\begin{figurehere}
\centering
\begin{tikzpicture}[
scale=1,
font=\bfseries\footnotesize\sffamily,
classical/.style={thick,<->,shorten >=2pt,shorten <=2pt,>=stealth},
oneway/.style={->,dashed,shorten >=2pt,shorten <=2pt,>=stealth}
]
    % Draw axes
    \draw [->,thick] (0,1) node (yaxis) [above] {$$}
        |- (6.2,0) node (xaxis) [right] {$$};
        
    \draw
  	(4,0) coordinate (A)
  	(4,1) coordinate (A2)
  	(4.8,-0.6) coordinate (B)
  	(4.8,1) coordinate (B2)
  	(6,-0.6) coordinate (C)
  	(6,1) coordinate (C2);
  	
  	\fill [draw=none, fill=gray!20] 
    (4.8, 0) rectangle (6, 1);
    
  	\fill [draw=none, fill=gray!10] 
    (0, -0.6) rectangle (4.8, 0);
	\draw[thick] (0, -0.6) -- (0, 0.6) node[below]{$$};
  	\draw[thick, thin] (A) -- (A2) node[below]{$$};
  	\draw[thick, thin] (B) -- (B2) node[below]{$$};
  	\draw[thick] (C) node[below, xshift=0.5cm]{$TotalMontoCompra$} -- (C2) ;
  	
  	\draw[classical] (0, 0.5) -> (4, 0.5) node[below]{$$};
  	\draw[classical] (4, 0.75) -> (4.8, 0.75) node[below]{$$};
%  	\draw[classical] (4.8, 0.5) -> (6, 0.5) node[below]{$$};
  	\draw[classical] (4, 0.25) -> (6, 0.25) node[below]{$$};


  	\draw[oneway] (2, 1.2) node[above]{$OrdenOriginalCompraMonto$} -- (2, 0.5);
  	\draw[oneway] (4.4, 2.2) node[above]{$CantidadCompraAdicional$} -- (4.4, 0.75);
  	\draw[oneway] (5.4, 1.6) node[above]{$MargenDivido$} -- (5.4, 1);
  	\draw[oneway] (5, -1.2) node[below]{$Margen$} -- (5, 0.25);
  	\draw[oneway] (2.4, -1.2) node[below]{$OrdenActualMontoCompra$} -- (2.4, -0.5);
\end{tikzpicture}
\caption{Un 60\% de Margen Dividido}
\label{fig:marginsplit}
\end{figurehere}
\end{center}
Si el margen en el anillo de pedidos es demasiado bajo, el minero de anillo escoger\'a la comisi\'on en LRx. Si, por lo contrario, el margen es lo suficientemente sustancial para que la divisi\'on del margen resultante valga mucho m\'as que la tarifa en LRx, un minero de anillo escoger\'a la divisi\'on del margen. Sin embargo, hay una condici\'on adicional: cuando el minero de anillo elige la divisi\'on de margen, debe pagarle al usuario (creador del pedido) una tarifa, que es igual al LRx que el usuario habr\'ia pagado al minero de anillo como tarifa. Esto aumenta el umbral de donde el minero de anillo elegir\'a compartir el margen para doblar la tarifa de LRx del pedido, aumentando as\'i la propensi\'on hacia la adopci\'on de la comisi\'on en LRx. Esto permite a los mineros de anillo obtener un ingreso constante en anillos de bajo margen, a cambio de recibir menos ingresos en pedidos con m\'argenes m\'as altos. Nuestro modelo de comisi\'on est\'a basado en la expectativa de que, a medida que el mercado crezca y madure, habr\'a un n\'umero cada vez menor de anillos de pedidos de margen alto, de ah\'i la necesidad de incentivar una comisi\'on fija en LRX.

Por lo tanto, obtenemos el siguiente gr\'afico: 

\begin{center}
\begin{figurehere}
\centering
\begin{tikzpicture}[
font=\bfseries\footnotesize\sffamily,
oneway/.style={->,dashed,shorten >=2pt,shorten <=2pt,>=stealth},
scale=1]
    % Draw axes
    \draw [<->,thick] (0,2.7) node (yaxis) [above] {$y$}
        |- (5,0) node (xaxis) [right] {$x$};
        
    \draw
  	(1,1) coordinate (A)
  	(2,1) coordinate (B);
  	
  	
  	\draw[thick] (B) -- (3.7,2.7);
  	\draw[dotted] (B) -- (2,0) node[below] {$2f$};
  	\draw[dotted] (A) -- (1,0) node[below] {$f$};
  	\draw[thick,color=gray!70] (0,0) -- (2.7,2.7);
  	\draw[thick] (0,1) node[left] {$f$}--(B) node[     ]{$$};
 	\draw[oneway] (4,1) node[right]{$IngresoDeMineriaEsperado$} -- (3, 2);
\end{tikzpicture}
\caption{Loopring's Fee Model}
\label{fig:feemodel}
\end{figurehere}
\end{center}

donde $f$ es la comisi\'on en LRx, $x$ es el margen dividido, $y$ es la comisi\'on de los mineros. $y=max(f, x-f)$ como lo indica la l\'inea solida; la comisi\'on en LRx para el pedido es $0$, la ecuaci\'on es  $y=max(0, x - 0)$ que se simplifica a $y=x$ como se indica en la l\'inea gris.

Las consecuencias son:
\begin{enumerate}
	\item 	S\'i el margen se divide en 0, los mineros de anillos escoger\'an la comisi\'on fija en LRx y a\'un son incentivados
	\item Si la comisi\'on en LRx es 0, el resultado es un modelo lineal gen\'erico representado por la l\'inea gris.
	\item Cuando el beneficio de la divisi\'on del margen es mayor a 2x (tarifa en LRx), los mineros elegir\'an la divisi\'on del margen, pagando la comisi\'on en LRx al usuario.
\end{enumerate}

Cabe se\~nalar que, si la comisi\'on en LRx es diferente de cero, no importa la opci\'on que el minero escoja, siempre habr\'a una transferencia de LRx entre el minero y el creador del pedido. O el minero gana la comisi\'on en LRx o paga la comisi\'on en LRx al remitente, para dividirse el margen.

Los mineros de anillo compartir\'an un cierto porcentaje de las tarifas con las carteras. Cuando un usuario realiza un pedido a trav\'es de una cartera y esta se ejecuta, el administrador de esta es compensado con una parte de la comisi\'on o la divisi\'on del margen. Aunque esto es modular,
y los modelos e implementaciones de negocios \'unicos sean posibles, consideramos que el porcentaje de comisiones que se debe compartir con las carteras es de aproximadamente un 20\% - 25\%. Las carteras son un objetivo clave para la implementaci\'on del protocolo de Loopring, ya que tienen su propia base de usuarios, pero poco o nada de fuentes de ingresos.


\subsection{Gobernanza Descentralizada}
En su esencia, el protocolo de Loopring se basa en un protocolo social, en el sentido de que depende de la coordinaci\'on de diferentes miembros, para permitirles colaborar de manera eficiente hacia un objetivo en com\'un. Esto no difiere de otros protocolos que caracterizan la criptoeconom\'ia en el sentido m\'as amplio, cuya utilidad est\'a regulada en gran parte por los mismos mecanismos de problemas de coordinaci\'on \cite{vitalikgovernance}, equilibrio de activaci\'on (grim trigger equilibrium) y racionalidad limitada. Con este fin, los tokens LRx no solo est\'an destinados al pago de tarifas, sino tambi\'en a alinear los incentivos financieros de los diversos participantes en la red. Esta alineaci\'on es una condici\'on fundamental para la adopci\'on de cualquier protocolo, pero a\'un m\'as para los protocolos de intercambio; considerando que el \'exito de este \'ultimo est\'a determinado por la capacidad de mejorar la liquidez en un ecosistema descentralizado. 

Los tokens LRx se usar\'an para efectuar actualizaciones de protocolo a trav\'es de una gobernanza descentralizada. Las actualizaciones de los contratos inteligentes, en parte, se regir\'an por los propietarios de los tokens (holders) para garantizar la continuidad y la seguridad, y para mitigar los riesgos de liquidez excesiva/sifonada causadas por los problemas de incompatibilidad. Dado que los contratos inteligentes no pueden ser modificados despu\'es de la implementaci\'on, existe el riesgo de que las dApps o los usuarios contin\'uen interactuando con las versiones desactualizadas, excluy\'endose del uso de contratos inteligentes actualizados. La capacidad de actualizaci\'on es crucial para el \'exito del protocolo, ya que se debe adaptar a las demandas del mercado y las blockchains subyacentes. Una gobernanza descentralizada de los titulares de LRx permitir\'a las actualizaciones de contratos inteligentes de protocolo, sin interrumpir a los dApps ni a los usuarios finales, o depender demasiado de la abstracci\'on de contratos inteligentes. Los tokens LRx existen en cantidades limitadas; y, en el caso de los LRC, un porcentaje de estos se mantienen congelados por la Fundaci\'on Loopring y son asignados como fondos para la comunidad \cite{LRCtokendoc}.

Sin embargo, los propietarios de los tokens LRx no son los \'unicos interesados a considerar en dirigir la direcci\'on del protocolo: los rel\'es/mineros de anillo, carteras, desarrolladores y otros son una parte integral del ecosistema y su voz debe ser escuchada. De hecho, dado que estos agentes no tienen la necesidad de poseer ning\'un LRx, para llevar a cabo su trabajo respectivo (dado que los creadores/tomadores y creadores de mercado tradicionales son inexistentes, las reservas de token iniciales no son obligatorias), tenemos que permitir m\'etodos alternativos para respetar sus intereses. Adem\'as, la votaci\'on \enquote{simple} basada en los tokens, tanto en cadena como fuera de ella, es un b\'alsamo imperfecto para el desacuerdo; ya que la baja participaci\'on de votantes y la concentraci\'on de propiedad de token representan riesgos. Por lo tanto, el objetivo es implementar un modelo de gobernanza que sea construido en capas y se base en un conocimiento compartido, de que un conjunto de procesos de toma de decisiones es la norma. Esto puede ser facilitado por instituciones de coordinaci\'on que ofrecen se\~nales de un conjunto diverso de participantes y, quiz\'as, de puntos focales de protocolo preestablecidos. Cuando esto llegue a buen t\'ermino de realizaci\'on, la Fundaci\'on de Loopring inevitablemente evolucionar\'a de desarrolladores de protocolos a administradores de protocolo.


\section{Protecciones Contra Ataques y Fraudes}
\subsection{Prevenci\'on del front-running\label{sec:dual_authoring}}
En las plataformas de intercambio descentralizado, la ejecuci\'on anticipada/front-running se produce cuando alguien intenta copiar la soluci\'on de intercambio de otro nodo, y la extrae en el bloque correspondiente antes de la transacci\'on original, que est\'a en el grupo de transacciones pendientes (mempool). Esta acci\'on se puede lograr ofreciendo una tarifa de transacci\'on m\'as alta (precio de gas). El principal esquema de \enquote{ejecuci\'on anticipada} o front-running en Loopring (y en cualquier protocolo de coincidencia-de-pedidos) es el robo de pedidos (order-filch): cuando un front-runner se roba uno o m\'as pedidos de una transacci\'on pendiente de liquidaci\'on del anillo de pedidos; y en espec\'ifico para Loopring, cuando un front-runner se roba el anillo completo de una transacci\'on pendiente.


Cuando una transacci\'on de tipo submittRing no ha sido confirmada y todav\'ia est\'a en el grupo de transacciones pendientes, cualquiera puede localizar f\'acilmente estas transacciones y reemplazar la direcci\'on del minero (\verb|minerAddress|) con su propia \enquote{direcci\'on de ladr\'on} (\verb|filcherAddress|); de esta manera, ellos pueden volver a firmar el paquete de datos con su \verb|filcherAddress|, para reemplazar la firma del anillo del pedido. El ladr\'on puede establecer un precio de gas m\'as alto y enviar una nueva transacci\'on, con la esperanza de que los mineros de bloques escojan su nueva transacci\'on dentro del siguiente bloque, en lugar de la transacci\'on original de submitRing.

Las soluciones anteriores para este problema ten\'ian desventajas considerables: requer\'ian m\'as transacciones y, por lo tanto, aumentaban los costos de gas para los mineros; tomando al menos dos veces la cantidad de bloques requeridos para asegurar un anillo de pedidos. Nuestra nueva soluci\'on, \enquote{Doble autor\'ia} (Dual Authoring) \cite{dualauthor}, que se compone de la adopci\'on de un mecanismo, que desarrolla dos niveles de autorizaci\'on para los pedidos: una para la regulaci\'on y otra para la miner\'ia de anillo.


Proceso de Dual Authoring:
\begin{enumerate}
	\item Para cada pedido, el software de la cartera generar\'a un par de llaves, llave p\'ublica/llave privada escogidas al azar, y colocar\'a ese par de llaves en una parte de JavaScript Object Notation (JSON) del pedido. (Una alternativa es usar la direcci\'on derivada de la llave p\'ublica en lugar de la llave p\'ublica misma para reducir el n\'umero de bytes requeridos. Usamos la \verb|authAddr| (direcci\'on de autorizaci\'on) para representar esta direcci\'on y \verb|authKey| (llave de autorizaci\'on) para presentar la llave privada correspondiente a \verb|authAddr|).
    \item Calcula el hash del pedido con todos los campos en el mismo pedido, excepto \verb|r|, \verb|v|, \verb|s| y \verb|authKey|, y firme el hash usando la llave privada del propietario (no la llave de autorizaci\'on \verb|authKey|).

    \item La cartera env\'ia el pedido junto a la llave de \verb|authKey| a los rel\'es para ser extra\'idos. Los mineros de los anillos verificar\'an que la llave de autentificaci\'on (\verb|authKey|) y la direcci\'on de autorizaci\'on (\verb|authAddr|) coincidan, y que la firma del pedido sea v\'alida en relaci\'on a la direcci\'on del \verb|propietario|. %\verb|owner|%


	\item Cuando un anillo de pedidos es identificado, el minero del anillo usa cada llave de autentificaci\'on \verb|authKey| de los pedidos para firmar el hash del anillo, direcci\'on del minero \verb|minerAddress|, y todos los par\'ametros de la miner\'ia. Si el anillo de pedidos contiene $n$ pedidos, habr\'an $n$ firmas para $n$ llaves de autentificaci\'on \verb|authKey|s. Llamamos a estas firmas, \verb|authSignature| (firmas de autorizaci\'on). El minero de anillos tambi\'en puede necesitar firmar el hash del anillo junto a todos los par\'ametros de la miner\'ia usando la llave privada de la direcci\'on del minero  \verb|minerAddress|.

	\item The ring-miner llama a la funci\'on submitRing con todos sus par\'ametros, as\'i tambi\'en como a las firmas de autentificaci\'on \verb|authSignature| adicionales. Tenga en cuenta que las llaves de autentificaci\'on \verb|authKey| NO son parte de la transacci\'on en la cadena y, por lo tanto, siguen siendo desconocidas para los participantes que no sean mineros de anillos (ring-miners).

	\item El Protocolo de Loopring, ahora, verificar\'a cada firma de autentificaci\'on \verb|authSignature| con la direcci\'on de autentificaci\'on \verb|authAddr| correspondiente, y rechazar\'a al anillo de pedidos si faltan cualquiera de sus firmas de \verb|authSignature| o si es que estas son invalidas.

\end{enumerate}
El resultado ahora es:
\begin{itemize}
    \item La firma del pedido (a trav\'es de la llave privada de la direcci\'on del \verb|propietario|) garantiza que dicho pedido no pueda ser modificado, incluyendo la \verb|authAddr|.
    \item La firma del \textit{minero del anillo}/ ring-miner (a trav\'es de la llave privada de la direcci\'on del minero \verb|minerAddress|), si es proporcionada, garantiza que nadie pueda usar su identidad para extraer un anillo de pedidos.
    \item  La firma \verb|authSignature| garantiza que no se pueda cambiar todo el anillo de pedidos, incluyendo la direcci\'on del minero \verb|minerAddress|, as\'i commo ning\'un pedido puede ser robado.
\end{itemize}


La Doble Autor\'ia (Dual Authoring, en ingl\'es) previene el robo de anillos y pedidos, y al mismo tiempo garantiza que la liquidaci\'on de los anillos de pedidos se pueda realizar en una sola transacci\'on. Adem\'as, Dual Authoring brinda la oportunidad para que los rel\'es compartan pedidos de dos maneras: compartici\'on no compatible y compartici\'on compatible. De manera predeterminada, Loopring opera siguiendo un modelo de OTC y solo permite pedidos con l\'imite de precio; lo que significa que la hora y la fecha del pedido son ignoradas. Esto implica que la front-running no tiene impacto en el precio registrado, pero si tiene impacto en decidir si el pedido es o no es ejecutado.


\section{Otros ataques}
\subsection{Ataque de tipo Sybil o Denial of Service (DOS)}
Los usuarios malintencionados -- actuando como ellos mismos o usando una identidad falsa -- podr\'ian enviar una gran cantidad de pedidos peque\~nos, para atacar y tratar de saturar los nodos de Loopring. Pero dado que el protocolo permite que los nodos acepten o rechacen pedidos, de acuerdo con su propio criterio -- criterio que puede ser escondido o revelado --, la mayor\'ia de estas ser\'ian rechazadas por falta de ganancia cuando sean emparejadas. Al permitir que los rel\'es administren pedidos como mejor les parezca, un ataque de este tipo no se considera una amenaza para el Protocolo de Loopring.

\subsection{Saldo insuficiente}
Los usuarios malintencionados podr\'ian armar y propagar pedidos cuyo valor no sea cero, pero cuyas direcciones tengan un saldo de cero. Los nodos podr\'ian monitorear y darse cuenta de que los saldos de estos pedidos son de cero, y simplemente actualizar el estado de estos, para deshacerse de ellos. Los nodos deben de dedicar algo de tiempo en actualizar el estado de un pedido, pero tambi\'en pueden elegir minimizar el esfuerzo de, por ejemplo, crear una lista direcciones negras e ignorar todos los pedidos asociados.

\section{Sumario}
El protocolo Loopring se propone ser una capa fundamental para las plataformas de intercambio descentralizado. Al hacerlo, el protocolo tiene un impacto profundo en c\'omo las personas intercambian sus activos y valores. El dinero, como un producto intermedio, facilita el intercambio tipo trueque y soluciona los problemas de la doble coincidencia de necesidades \cite{unenumerated2006}, donde dos contrapartes deben necesitar mutuamente los bienes o servicios que la otra parte ofrece. De forma similar, el protocolo Loopring tiene como prop\'osito eliminar nuestras dependencias de coincidencia de necesidades de los pares de intercambios comerciales, usando el anillo de coincidencia para realizar operaciones de intercambio con m\'as facilidad. Esto es importante para la forma en la que la sociedad y los mercados intercambian tokens, activos tradicionales y otros. De hecho, como las criptomonedas descentralizadas amenazan el control que una naci\'on ejerce sobre el dinero, un protocolo combinatorio que puede conectar a las partes que desean intercambiar (consumidores/productores) a gran escala, es te\'oricamente una amenaza para el concepto mismo del dinero.






Los principales beneficios del protocolo son:
\begin{itemize}
	\item Una gesti\'on de pedidos fuera de cadena y liquidaci\'on en cadena; significa que no se sacrificara el rendimiento por la seguridad..
	\item Una mejora de la liquidez mediante la extracci\'on de anillos y el intercambio de pedidos.
	\item Una Dual Authoring, que consta de dos niveles de autorizaci\'on para los pedidos, resuelve el problema da\~nino de la front-running; problema que es afrontado hoy en d\'ia por todas las Plataformas de Intercambio Descentralizado (DEXs) y sus usuarios.
	\item Unos contratos inteligentes p\'ublicos gratuitos, que permiten a cualquier App descentralizada (dApp) construir o interactuar con el protocolo.
	\item Una estandarizaci\'on entre operadores permite efectos de red y una mejor experiencia para el usuario final.
	\item Una red mantenida con flexibilidad en la ejecuci\'on de los libros de pedido y comunicaci\'on.
	\item Unas barreras de entrada reducidas significan costos m\'as bajos para los nodos que se unen a la red y para los usuarios finales.
	\item Un intercambio an\'onimo hecho directamente desde las carteras de los usuarios.	
\end{itemize}


\section{Reconocimientos}
Nos gustar\'ia expresar nuestra gratitud a nuestros mentores, asesores y a muchas personas de la comunidad que han sido tan acogedoras y generosas con su conocimiento. En particular, nos gustar\'ia agradecer a Shuo Bai (de ChinaLedger); al Profesor Haibin Kan; Alex Cheng, Hongfei Da; Yin Cao; Xiaochuan Wu; Zhen Wang, Wei Yu, Nian Duan, Jun Xiao, Jiang Qian, Jiangxu Xiang, Yipeng Guo, Dahai Li, Kelvin Long, Huaxia Xia, Jun Ma, y a Encephalo Path por examinar y aconsejarnos en este proyecto mediante sus comentarios.



\bibliography{whitepaper}
\bibliographystyle{unsrt}
\end{multicols}


%\begin{appendices}
%\section{Loopring Implemented on EVM\label{app:protocol_ethereum}}
%\begin{center}
%\begin{figurehere}
%\centering
%\begin{tikzpicture}
%[node distance = 1cm, auto,font=\footnotesize,
% STYLES
%every node/.style={node distance=3cm},
% The comment style is used to describe the characteristics of each force
%comment/.style={rectangle, inner sep= 5pt, text width=4cm, node distance=0.25cm, font=\scriptsize\sffamily},
% The force style is used to draw the forces' name
%force/.style={rectangle, draw, fill=black!10, inner sep=5pt, text width=4cm, text badly centered, minimum height=1.2cm, font=\bfseries\footnotesize\sffamily}] 
% Draw forces
%\node [force] (impl) {LoopringProtocolImpl};
%\node [force, dashed, above of=impl] (protocol_interface) {LoopringProtocol};
%\node [force, left=1cm of impl] (nameregistry) {NameRegistry};
%\node [force, right=1cm of impl] (tokenregistry) {TokenRegistry};
%\node [force, below of=impl] (delegate) {TokenTransferDelegate};
%\node [force, left=1cm of delegate] (multisig) {TransferableMultsig};
%%%%%%%%%%%%%%%

% Change data from here
% impl
%\node [comment, below=0.25 of impl] (comment-impl) {- Validates order-rings\\
%- Transfers tokens for settlement\\
%- Emits events};
% nameregistry
%\node [comment, below=0.25cm of nameregistry]{- Registers wallets and relays};
% protocol_interface
%\node [comment, below=0.25 of protocol_interface](comment-interface) {- Defines interfaces and events};
% tokenregistry
%\node [comment, below=0.25 of tokenregistry] {- Registers ERC20/ERC223 tokens};
% delegate
%\node [comment, below=0.25 of delegate] {- Transfers tokens on behalf of users};
% PUBLIC POLICIES
%\node [comment, text width=3cm, below=0.25 of multisig] {- Enables multisignature ownership};
%%%%%%%%%%%%%%%%
% Draw the links between forces
%\path[->,thick] 
%(comment-interface) edge (impl)
%(nameregistry) edge (impl)
%(tokenregistry) edge (impl)
%(delegate) edge (comment-impl);
%\end{tikzpicture} 
%\caption{Smart Contracts}
%\label{fig:smartcontracts}
%\end{figurehere}
%\end{center}
%\section{Deployment}
%\subsection{Ethereum}
%The following smart contracts have been deployed on Ethereum mainnet:
%\begin{itemize}
%\item LRC: \verb|0xEF68e7C694F40c8202821eDF525dE3782458639f|
%\item TokenRegistry: \verb|0xa21c1f2AE7f721aE77b1204A4f0811c642638da9|
%\item TokenTransferDelegate: \verb|0x7b126ab811f278f288bf1d62d47334351dA20d1d|
%\item NameRegistry: \verb|0xd181c1808e3f010F0F0aABc6Fe1bcE2025DB7Bb7|
%\item LoopringProtocolImpl: \verb|0x0B48b747436f10c846696e889e66425e05CD740f|
%\end{itemize}
%\subsection{Qtum}
%The following smart contracts have been deployed on Qtum mainnet:
%\begin{itemize}
%\item LRQ: \verb| 2eb2a66afd4e465fb06d8b71f30fb1b93e18788d |
%\item TokenRegistry: \verb| c89ea34360258917daf3655f8bec5550923509b3 |
%\item TokenTransferDelegate: \verb| 60b3fa7f461664e4dafb621a36ac2722cc680f10 |
%\item NameRegistry: \verb| e26a27d92181069b25bc7283e03722f6ce7678bb |
%\item LoopringProtocolImpl: \verb| 5180bb56b696d16635abd8dc235e0ee432abf25d |
%\end{itemize}
%\end{appendices}
\end{document}