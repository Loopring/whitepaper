% !TEX program = XeLaTeX
% !TEX encoding = UTF-8
\documentclass[UTF8,nofonts]{ctexart}
\setCJKmainfont[BoldFont=FandolSong-Bold.otf,ItalicFont=FandolKai-Regular.otf]{FandolSong-Regular.otf}
\setCJKsansfont[BoldFont=FandolHei-Bold.otf]{FandolHei-Regular.otf}
\setCJKmonofont{FandolFang-Regular.otf}

\usepackage{url}
\usepackage{cancel}
\usepackage{xspace}
\usepackage{graphicx}
\usepackage{multicol}
\usepackage{multirow}
\usepackage{subfig}
\usepackage{amsmath}
\usepackage{amssymb}
%\usepackage[a4paper,width=180mm,top=18mm,bottom=22mm,includeheadfoot]{geometry}
\usepackage[a4paper,width=140mm,top=18mm,bottom=22mm,includeheadfoot]{geometry}
\usepackage{booktabs}
\usepackage{array}
\usepackage{verbatim}
\usepackage{caption}
\usepackage{natbib}
\usepackage{booktabs}
\usepackage{float}
\usepackage{pdflscape}
\usepackage{mathtools}
\usepackage[usenames,dvipsnames]{xcolor}
\usepackage{afterpage}
\usepackage{pgf}
\usepackage{tikz}
\usepackage{dirtree}
\usepackage{amsfonts}
\usepackage{tkz-graph}
\newtheorem{definition}{定义}[section]
\newtheorem{theorem}{定理}[section]
\newtheorem{lemma}{Lemma}
\newtheorem{proof}{证明} [section]
\usepackage[toc,page,title,titletoc,header]{appendix}
\usepackage{marginnote}
\usepackage{tablefootnote}

\renewcommand\appendixname{附\ 录}
\renewcommand\appendixpagename{附\ 录}
\renewcommand\appendixtocname{附\ 录}

\usepackage{perpage} %the perpage package
\MakePerPage{footnote} %the perpage package command

\usetikzlibrary{shapes.geometric}%
\usepackage{color}
%\usepackage[pages=some,placement=top]{background}
\usepackage{eso-pic}

\title{\textbf{TokenBank}\\v0.1}
\author{
    daniel@TokenBank.com
}

\makeatletter
\def\CTEX@section@format{\Large\bfseries}
\makeatother

\makeatletter
\newenvironment{tablehere}
  {\def\@captype{table}}
  {}

\newenvironment{figurehere}
  {\def\@captype{figure}}
  {}
\makeatother


\begin{document}
%\AddToShipoutPicture{\BackgroundPic}
\maketitle

\section{简述\label{sec:idea}}

我将TokenBank定位成一个企业级的非托管钱包。它适用于企业级用户,大额用户,以及拥有多地址,多代币的虚拟资产持有者。我们和市面上现有的钱包软件几乎没有任何竞争关系。这样的钱包不但适合企业用户,也同样适用个人用户。是中心化和去中心化技术最好的结合,能够提供最佳的用户体验。我已经注册了tokenbank.com域名。

\section{目标问题\label{sec:idea}}

\subsection{转账历史维护困难}
对于经常做区块链转账的人来讲,很难追踪每笔tx都花在了什么地方,每笔收入的来源到底是哪个人或实体。目前最好的方法就是用excel记录下来。TokenBank会要求用户在做转账的时候选择预定义的一些类别,或者加上自己的注释。这些记录会和tx id放到TokenBank中心数据库中,对企业内部(部分)管理者可见。

另外我们对经常使用的地址,也可以像微信中那也,加tag或者label,这样后续使用更不容易出错。

\subsection{多地址、多币种难管理}

越来越多的人,特别是机构,慢慢会拥有大量的地址,这些地址的管理很成问题。现在市面上所有的钱包对多个地址的管理都存在问题。比如以太坊官方go语言钱包,如果地址超过50个,就会极其缓慢,而且地址的名称无法很好管理,把keystore 文件导出后,地址名字完全丢失。

TokenBank会将所有地址在TokenBank的智能合约里关联起来。这样可以通过单一的入口地址(root address)对所有地址进行管理。核心管理者可以授权新的私钥一定的花费额度,并且通过App可以授权和查看所有的花费情况(需要多重签名支持)。

\subsection{多重签名不好用}
对于机构用户,多重签名十分重要,但多重签名的流程如果没有中心和服务器的接入,很难提供好的用户体验。TokenBank通过多重签名流程的中心化管理,可以将这个过程极大简化,提供更好的用户体验。

\subsection{缺乏资产灾难管理}

TokenBank允许用户通过智能合约,设置遗产分配。比如可以设置多个账号按照一定比例分割token和ETH。一旦该root账号超过指定的时间没有活跃(没有进行任何的私钥签名),遗产继承账号就可以通过智能合约claim应得的资产。

\subsection{机构资产的交易依然需要信任交易所}
我们将允许用户通过使用Loopring协议进行交易,特别是大额,匿名的交易。

\subsection{很难批量进行转账}

现在如果为几十个用户发ETH工资,是个很痛苦的过程。TokenBank可以帮助用户生成多个tx,一键签名即可为所有交易授权。如果转账不成功,也会主动提醒再次授权新的交易。生成多个tx时候可以指定是以法币为价值标准,自动按照当前价格换算成ETH或者特定的TOKEN,也可以指定固定数量的ETH或者token。

\subsection{转账通知}
结合email和电话,可以对转账成功或失败进行通知,因为有了各种标注,通知的内容更加直观。

\subsection{转入和转出Hook}
企业管理员可以根据特定条件,设置到账和出账后自动执行一些操作。这个功能主要用作将TokenBank做特定功能的拓展,便于第三方程序员的个性化。

\subsection{缺乏企业转账的报表、审计、和税务管理}

TokenBank还会提供转账记录的导出功能,用于企业后续的记账。所有记录都会根据企业进行汇总,并且标注是哪个人进行的转账,当时的USD/CNY价格是多少,等等。各种数据可以通过web下载,生成特定的报表。

以后会根据地区和国家法律,做税务报表汇总和生成等。

\section{内置代币}

我们可以设计内置代币,也可完全股权融资。TokenBank的内置代币可以为上面所有服务进行计费。

\end{document}
